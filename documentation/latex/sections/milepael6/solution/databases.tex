\subsection{Nøkkel-verdi}
Nøkkel-verdi databasen serverer brukeren med raske queries og horizontalt skalerbar arkitektur som tillater at å legge inn nye skoler med ny data blir enkelt. Dette er noe som kunne vært fint å bruke for et land eller fylke som ønsker å følge utdanningsnivåer og socio-økonomiske tilstander, i tillegg til administratorer o.l. ved skoler som ønsker å følge nasjonale eller internasjonale trender og kvalitet. Siden det heller ikke er krav om satt struktur, så kan ny data legges inn så og si slik den er.

\subsection{DokumentDb}
I dokument databasen er det landsprofiler med mye tilhørende data som vil endre seg relativt ofte. Disse profilene vil endre seg relativt ofte, og krever da høy fleksibilitet. I tillegg er det forskjellig informasjon som vil være nødvendig og tilgjengelig på forskjellige profiler, så en skjema-løs ordning er bra. En annen fordel er hvordan data fordeles, og gjør at nøstede og dypere søk yter bedre, og skaper høyere tilgjengelighet for brukeren.

\subsection{Kolonnefamilie}
I Kolonnefamilie-databasen har vi mange ferdige aggregeringer. Disse aggregeringene skal være lett tilgjengelige, og de skal kunne lagres i forskjellige kolonner innen samme tabell. Siden denne db-modellen også er skjemaløst, så kan vi sette kolonnenavn som vi vil, også innen samme tabell, og vi kan legge til nye kolonner i real-time om nødvendig. Med mange aggregeringer og mye data lagret i en kolonne, så reduserer det de nødvendige ressursene fra disken og hvor lang tid spørringene tar.

\subsection{Grafdatabase}
Grafdatabasen har vi dypt relatert data som brukeren skal kunne lett se sammenhengen i. En bruker som følger med på internasjonal politikk eller verdensutvikling kan bruke dette for eksempel til å se styrken til regimer, hvilke land som deler politisk sammensetning, og deres stabilitet. Det som er viktig med denne seksjonen er at brukeren lett kan se sammenhengene, f.eks. kan man med en graphdb hente ut data basert på koblinger(edges), og ikke bare nodene.