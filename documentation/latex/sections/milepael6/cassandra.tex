\subsection{Cassandra}

\subsubsection{SAN vs NAS}
Kolonnefamiliedatabasen Cassandra lagrer dataen i et filsystem kalt for \textit{shared-disk file system} som baserer seg for et \textit{Storage Area Network}(SAN) system. \textit{Network-Attached System}(NAS) derimot, er lettere å sette opp og annses på som et hjemmenettverk. Det er et filsystem som tilbyr fleksible lagringsmuligheter og er enkelt å sette opp, som er fint siden det ikke krever spesiell kompetanse innen IT.

Der SAN skinner er at det er høy-ytelse og lav ventetid. Det er vanskeligere å sette opp en NAS, men siden vi leverer tjenesten har ikke dette noe å si for brukergruppen. Vi setter mye større prioritet på ventetiden, siden denne vokser vesentlig med mendgen data. Siden vi prioriterer tilgjengelighet i systemet hjelper det mye at SAN forminsker stress i et area-network, på den måten hindrer det nettverksfeil.

\subsubsection{Kolonnefamiliedatabase}
Cassandra er verktøyet vi skulle bruke for å opprette Kolonnefamiliedatabaser. Fra tidligere hadde vi allerede designet aggregeringene og laget de via spark-shell i Scala, så prossessen var heldigvis rett fram. For alle aggregeringene, inkluderende rådataen, måtte man opprette en ny table/kolonnefamilie i Cassandra med samsvarende kolonnenavn og datatype.

\subsubsection{Laste opp rådata}
Første dataframe som skal lastes opp til Cassandra er rådataen. Dette er grunnlaget for all data som skal vises på nettsiden, og brukes når man skal legge til eller slette rader fra datasettet, slik at alle de andre komponentene også blir oppdatert. Her har vi valgt å inkludere alle kolonnene i datasettet, selv om mange av de ikke brukes av komponentene. Det skyldes at vi ønsker å kunne laste opp hele datasettet fra kilden igjen dersom det skulle oppdatere seg, samt at vi ikke har lyst til å kaste bort data. I Big Data pleier man å lagre all data, siden det kan komme til nytte senere.

\code{Scala}{code/milepael6/populateCassandraRaw.scala}

Når timesData.csv leses inn i Spark-Shell, brukes \lstinline{inferSchema} til å automatisk generere skjema basert på dataen i datasettet. Her er det bare viktig å passe på at alle datatypene stemmer, siden Spark ofte gjetter feil. I tillegg brukes \lstinline{confirm.truncate} og \lstinline{Overwrite} til å overskrive eksisterende data i Cassandra. Dersom det er ønskelig å laste opp rad for rad, kan \lstinline{Append} brukes i stedet. Til slutt lastes inn datasettet fra Cassandra som blir brukt til de resterende komponentene.

\subsubsection{Aggregeringer til kjønnsfordeling på skoler}
Alle aggregeringene opprettes og bearbeides i spark-shell før de blir lastet opp til Cassandra. Fra tidligere kodesnutt brukes \lstinline{newUniversityDf} til å lage aggregeringene, men dette er helt vilkårlig, det går fint ann å jobbe direkte på den opprinnelige dataframen.

Legg merke til bruken av \lstinline{orderBy}. I designet til nettsiden er det tenkt at komponenten skal vise top ti skoler med høyest kvinnefordeling sortert i synkende rekkefølge, men etter å ha laget denne aggregeringen oppsto det et problem. Det viser seg at kolonnefamiliedatabaser ikke tar hensyn til sortering, bortsett fra at de fremstilles i alfabetisk rekkefølge. På grunn av dette har ikke \lstinline{orderBy} eller \lstinline{sort} noen effekt etter at aggregeringen har blitt lastet opp til Cassandra. Dette kan imidlertid løses ved at enten spark eller businesslogikken tar hensyn til sorteringen når den vises fram på nettsiden.

\code{Scala}{code/milepael6/populateCassandraFemaleRatio.scala}

Tilsvarende aggregering for menn:

\code{Scala}{code/milepael6/populateCassandraMaleRatio.scala}

\subsubsection{Gjennomsnittlig poengsum hvert år}
Den siste aggregeringen tar for seg den gjennomsnittlige totalpoengsummen hvert år i datasettet. Vært å merke er at det gjøres ingen handlinger på selve aggregeringen før den lastes opp til databasen, da \lstinline{averageSorePerYear} lagrer kun en liste over transformasjoner som skal gjøres på dataframen. Dersom variabelen hadde inkludert \lstinline{.show()}, som er en handling, ville hver enkel transformasjon eksekveres etter hverandre.

\code{Scala}{code/milepael6/populateCassandraAverageScorePerYear.scala}

\subsubsection{Hvordan nettsiden henter data}
Det finnes mange måter å håndtere uthenting på, men siden nettsiden må kunne lese dataen, var det logisk å gjøre om til \textit{Json} format. I ettertid ser vi at det hadde vært mer logisk å lage et eget Json objekt i Scala, men istedet lagres de direkte som Json filer i systemet.

\code{Scala}{code/milepael6/websiteCassandra.scala}