\subsection{SAN vs NAS}
Kolonnefamiliedatabasen Cassandra lagrer dataen i et filsystem kalt for \textit{shared-disk file system} som baserer seg for et \textit{Storage Area Network}(SAN) system. \textit{Network-Attached System}(NAS) derimot, er lettere å sette opp og annses på som et hjemmenettverk. Det er et filsystem som tilbyr fleksible lagringsmuligheter og er enkelt å sette opp, som er fint siden det ikke krever spesiell kompetanse innen IT.

Der SAN skinner er at det er høy-ytelse og lav ventetid. Det er vanskeligere å sette opp en NAS, men siden vi leverer tjenesten har ikke dette noe å si for brukergruppen. Vi setter mye større prioritet på ventetiden, siden denne vokser vesentlig med mendgen data. Siden vi prioriterer tilgjengelighet i systemet hjelper det mye at SAN forminsker stress i et area-network, på den måten hindrer det nettverksfeil. 